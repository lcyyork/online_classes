%%%%%%%%%%%%%%%%%%%%%%%%%%%%%%%%%%%%%%%%%
% Structured General Purpose Assignment
% LaTeX Template
%
% This template has been downloaded from:
% http://www.latextemplates.com
%
% Original author:
% Ted Pavlic (http://www.tedpavlic.com)
%
% Note:
% The \lipsum[#] commands throughout this template generate dummy text
% to fill the template out. These commands should all be removed when 
% writing assignment content.
%
%%%%%%%%%%%%%%%%%%%%%%%%%%%%%%%%%%%%%%%%%

%----------------------------------------------------------------------------------------
%	PACKAGES AND OTHER DOCUMENT CONFIGURATIONS
%----------------------------------------------------------------------------------------

\documentclass{article}

\usepackage{fancyhdr} % Required for custom headers
\usepackage{lastpage} % Required to determine the last page for the footer
\usepackage{extramarks} % Required for headers and footers
\usepackage{graphicx} % Required to insert images
\usepackage{lipsum} % Used for inserting dummy 'Lorem ipsum' text into the template

% Margins
\topmargin=-0.45in
\evensidemargin=0in
\oddsidemargin=0in
\textwidth=6.5in
\textheight=9.0in
\headsep=0.25in 

\linespread{1.1} % Line spacing

% Set up the header and footer
\pagestyle{fancy}
%\lhead{\hmwkAuthorName} % Top left header
\lhead{\hmwkClass\ (\hmwkClassInstructor): \hmwkTitle} % Top center header
%\chead{\hmwkClass\ (\hmwkClassInstructor\ \hmwkClassTime): \hmwkTitle} % Top center header
\rhead{\firstxmark} % Top right header
\lfoot{\lastxmark} % Bottom left footer
\cfoot{} % Bottom center footer
\rfoot{Page\ \thepage\ of\ \pageref{LastPage}} % Bottom right footer
\renewcommand\headrulewidth{0.4pt} % Size of the header rule
\renewcommand\footrulewidth{0.4pt} % Size of the footer rule

\setlength\parindent{0pt} % Removes all indentation from paragraphs

%----------------------------------------------------------------------------------------
%	DOCUMENT STRUCTURE COMMANDS
%	Skip this unless you know what you're doing
%----------------------------------------------------------------------------------------

% Header and footer for when a page split occurs within a problem environment
\newcommand{\enterProblemHeader}[1]{
\nobreak\extramarks{#1}{#1 continued on next page\ldots}\nobreak
\nobreak\extramarks{#1 (continued)}{#1 continued on next page\ldots}\nobreak
}

% Header and footer for when a page split occurs between problem environments
\newcommand{\exitProblemHeader}[1]{
\nobreak\extramarks{#1 (continued)}{#1 continued on next page\ldots}\nobreak
\nobreak\extramarks{#1}{}\nobreak
}

\setcounter{secnumdepth}{0} % Removes default section numbers
\newcounter{homeworkProblemCounter} % Creates a counter to keep track of the number of problems

\newcommand{\homeworkProblemName}{}
\newenvironment{homeworkProblem}[1][Problem \arabic{homeworkProblemCounter}]{ % Makes a new environment called homeworkProblem which takes 1 argument (custom name) but the default is "Problem #"
\stepcounter{homeworkProblemCounter} % Increase counter for number of problems
\renewcommand{\homeworkProblemName}{#1} % Assign \homeworkProblemName the name of the problem
\section{\homeworkProblemName} % Make a section in the document with the custom problem count
\enterProblemHeader{\homeworkProblemName} % Header and footer within the environment
}{
\exitProblemHeader{\homeworkProblemName} % Header and footer after the environment
}

\newcommand{\problemAnswer}[1]{ % Defines the problem answer command with the content as the only argument
\noindent\framebox[\columnwidth][c]{\begin{minipage}{0.98\columnwidth}#1\end{minipage}} % Makes the box around the problem answer and puts the content inside
}

\newcommand{\homeworkSectionName}{}
\newenvironment{homeworkSection}[1]{ % New environment for sections within homework problems, takes 1 argument - the name of the section
\renewcommand{\homeworkSectionName}{#1} % Assign \homeworkSectionName to the name of the section from the environment argument
\subsubsection{\homeworkSectionName} % Make a subsection with the custom name of the subsection
\enterProblemHeader{\homeworkProblemName\ [\homeworkSectionName]} % Header and footer within the environment
}{
\enterProblemHeader{\homeworkProblemName} % Header and footer after the environment
}
   
%----------------------------------------------------------------------------------------
%	NAME AND CLASS SECTION
%----------------------------------------------------------------------------------------

\newcommand{\hmwkTitle}{Optional Theory Problems\ \#2} % Assignment title
%\newcommand{\hmwkDueDate}{Monday,\ January\ 1,\ 2012} % Due date
\newcommand{\hmwkClass}{CS\ 161} % Course/class
%\newcommand{\hmwkClassTime}{10:30am} % Class/lecture time
\newcommand{\hmwkClassInstructor}{Tim Roughgarden} % Teacher/lecturer
%\newcommand{\hmwkAuthorName}{John Smith} % Your name

%----------------------------------------------------------------------------------------
%	TITLE PAGE
%----------------------------------------------------------------------------------------

\title{
\vspace{2in}
\textmd{\textbf{\hmwkClass:\ \hmwkTitle}}\\
%\normalsize\vspace{0.1in}\small{Due\ on\ \hmwkDueDate}\\
\vspace{0.1in}\large{Instructor: \textit{\hmwkClassInstructor}}
%\vspace{0.1in}\large{\textit{\hmwkClassInstructor\ \hmwkClassTime}}
\vspace{3in}
}

%\author{\textbf{\hmwkAuthorName}}
\date{} % Insert date here if you want it to appear below your name

%----------------------------------------------------------------------------------------

\begin{document}

%\maketitle

%----------------------------------------------------------------------------------------
%	TABLE OF CONTENTS
%----------------------------------------------------------------------------------------

%\setcounter{tocdepth}{1} % Uncomment this line if you don't want subsections listed in the ToC

%\newpage
%\tableofcontents
%\newpage

%%%
%  General stuff
%%%

The following problems are for those of you looking to challenge yourself beyond the required problem sets and programming questions.
Most of these have been given in Stanford's CS161 course, Design and Analysis of Algorithms, at some point.
They are completely optional and will not be graded.
While they vary in level, many are pretty challenging, and we strongly encourage you to discuss ideas and approaches with your fellow students on the ``Theory Problems'' discussion form.

\begin{homeworkProblem}
In the 2SAT problem, you are given a set of clauses, where each clause is the disjunction of two literals (a literal is a Boolean variable or the negation of a Boolean variable).
You are looking for a way to assign a value ``true'' or ``false'' to each of the variables so that all clauses are satisfied --- that is, there is at least one true literal in each clause.
For this problem, design an algorithm that determines whether or not a given 2SAT instance has a satisfying assignment.
(Your algorithm does not need to exhibit a satisfying assignment, just decide whether or not one exists.)
Your algorithm should run in ${\cal O}(m+n)$ time, where $m$ and $n$ are the number of clauses and variables, respectively.
[Hint: strongly connected components.]
\end{homeworkProblem}

\begin{homeworkProblem}
In lecture we define the length of a path to be the sum of the lengths of its edges.
Define the bottleneck of a path to be the maximum length of one of its edges.
A \emph{mininum-bottleneck} path between two vertices $s$ and $t$ is a path with bottleneck no larger than that of any other $s$-$t$ path.

\textbf{(a)}
Show how to modify Dijkstra's algorithm to compute a minimum-bottleneck path between two given vertices.
The running time should be ${\cal O}(m \log n)$, as in lecture.

\textbf{(b)}
We can do better. Suppose now that the graph is undirected. Give a linear-time [${\cal O}(m)$] algorithm to compute a minimum-bottleneck path between two given vertices.

\textbf{(c)}
What if the graph is directed?
Can you compute a minimum-bottleneck path between two given vertices faster than ${\cal O}(m \log n)$?

\end{homeworkProblem}

\begin{homeworkProblem}
Recall that a set $H$ of hash functions (mapping the elements of a universe $U$ to the buckets $\{0, 1, 2, \cdots, n-1\}$) is universal if for every distinct $x, y \in U$, the probability ${\rm Prob}[h(x)=h(y)]$ that $x$ and $y$ collide, assuming that the hash function $h$ is chosen uniformly at random from $H$, is at most $1/n$.
In this problem you will prove that a collision probability of $1/n$ is essentially the best possible.
Precisely, suppose that $H$ is a family of hash functions mapping $U$ to $\{0, 1, 2, \cdots, n-1\}$, as above.
Show that there must be a pair $x, y \in U$ of distinct elements such that, if $h$ is chosen uniformly at random from $H$, then ${\rm Prob} [ h(x) = h(y) ] \geq \frac{1}{n} - \frac{1}{|U|}$.
\end{homeworkProblem}

%\begin{homeworkProblem}
%We showed in an optional video lecture that every undirected graph has only polynomially (in the number n of vertices) different minimum cuts. Is this also true for directed graphs? Prove it or give a counterexample.
%\end{homeworkProblem}
%
%\begin{homeworkProblem}
%For a parameter $\alpha \geq 1$, an $\alpha$-minimum cut is one for which the number of crossing edges is at most $\alpha$ times that of a minimum cut. How many $\alpha$-minimum cuts can an undirected graph have, as a function of $\alpha$ and the number $n$ of vertices? Prove the best upper bound that you can.
%\end{homeworkProblem}

%%----------------------------------------------------------------------------------------
%%	PROBLEM 1
%%----------------------------------------------------------------------------------------
%
%% To have just one problem per page, simply put a \clearpage after each problem
%
%\begin{homeworkProblem}
%\lipsum[1]\vspace{10pt} % Question
%
%\problemAnswer{ % Answer
%%\begin{center}
%%\includegraphics[width=0.75\columnwidth]{example_figure} % Example image
%%\end{center}
%
%\lipsum[2]
%}
%\end{homeworkProblem}
%
%%----------------------------------------------------------------------------------------
%%	PROBLEM 2
%%----------------------------------------------------------------------------------------
%
%\begin{homeworkProblem}[Exercise \#\arabic{homeworkProblemCounter}] % Custom section title
%\lipsum[3] % Question
%
%%--------------------------------------------
%
%\begin{homeworkSection}{(a)} % Section within problem
%\lipsum[4]\vspace{10pt} % Question
%
%\problemAnswer{ % Answer
%\lipsum[5]
%}
%\end{homeworkSection}
%
%%--------------------------------------------
%
%\begin{homeworkSection}{(b)} % Section within problem
%\problemAnswer{ % Answer
%\lipsum[6]
%}
%\end{homeworkSection}
%
%%--------------------------------------------
%
%\end{homeworkProblem}
%
%%----------------------------------------------------------------------------------------
%%	PROBLEM 3
%%----------------------------------------------------------------------------------------
%
%\begin{homeworkProblem}[Prob. \Roman{homeworkProblemCounter}] % Roman numerals
%
%%--------------------------------------------
%
%\begin{homeworkSection}{\homeworkProblemName:~(a)} % Using the problem name elsewhere
%\problemAnswer{ % Answer
%\lipsum[7]
%}
%\end{homeworkSection}
%
%%--------------------------------------------
%
%\begin{homeworkSection}{\homeworkProblemName:~(b)}
%\lipsum[8]\vspace{10pt} % Question
%
%\problemAnswer{ % Answer
%\lipsum[9]
%}
%\end{homeworkSection}
%
%%--------------------------------------------
%
%\end{homeworkProblem}
%
%%----------------------------------------------------------------------------------------
%%	PROBLEM 4
%%----------------------------------------------------------------------------------------
%
%\begin{homeworkProblem}[Prob. \Roman{homeworkProblemCounter}] % Roman numerals
%\problemAnswer{ % Answer
%\lipsum[10]
%}
%\end{homeworkProblem}

%----------------------------------------------------------------------------------------

\end{document}
