%%%%%%%%%%%%%%%%%%%%%%%%%%%%%%%%%%%%%%%%%
% Structured General Purpose Assignment
% LaTeX Template
%
% This template has been downloaded from:
% http://www.latextemplates.com
%
% Original author:
% Ted Pavlic (http://www.tedpavlic.com)
%
% Note:
% The \lipsum[#] commands throughout this template generate dummy text
% to fill the template out. These commands should all be removed when 
% writing assignment content.
%
%%%%%%%%%%%%%%%%%%%%%%%%%%%%%%%%%%%%%%%%%

%----------------------------------------------------------------------------------------
%	PACKAGES AND OTHER DOCUMENT CONFIGURATIONS
%----------------------------------------------------------------------------------------

\documentclass{article}

\usepackage{fancyhdr} % Required for custom headers
\usepackage{lastpage} % Required to determine the last page for the footer
\usepackage{extramarks} % Required for headers and footers
\usepackage{graphicx} % Required to insert images
\usepackage{lipsum} % Used for inserting dummy 'Lorem ipsum' text into the template
\usepackage{hyperref}
\usepackage{ams math}

% Margins
\topmargin=-0.45in
\evensidemargin=0in
\oddsidemargin=0in
\textwidth=6.5in
\textheight=9.0in
\headsep=0.25in 

\linespread{1.1} % Line spacing

% Set up the header and footer
\pagestyle{fancy}
%\lhead{\hmwkAuthorName} % Top left header
\lhead{\hmwkClass\ (\hmwkClassInstructor): \hmwkTitle} % Top center header
%\chead{\hmwkClass\ (\hmwkClassInstructor\ \hmwkClassTime): \hmwkTitle} % Top center header
\rhead{\firstxmark} % Top right header
\lfoot{\lastxmark} % Bottom left footer
\cfoot{} % Bottom center footer
\rfoot{Page\ \thepage\ of\ \pageref{LastPage}} % Bottom right footer
\renewcommand\headrulewidth{0.4pt} % Size of the header rule
\renewcommand\footrulewidth{0.4pt} % Size of the footer rule

\setlength\parindent{0pt} % Removes all indentation from paragraphs

%----------------------------------------------------------------------------------------
%	DOCUMENT STRUCTURE COMMANDS
%	Skip this unless you know what you're doing
%----------------------------------------------------------------------------------------

% Header and footer for when a page split occurs within a problem environment
\newcommand{\enterProblemHeader}[1]{
\nobreak\extramarks{#1}{#1 continued on next page\ldots}\nobreak
\nobreak\extramarks{#1 (continued)}{#1 continued on next page\ldots}\nobreak
}

% Header and footer for when a page split occurs between problem environments
\newcommand{\exitProblemHeader}[1]{
\nobreak\extramarks{#1 (continued)}{#1 continued on next page\ldots}\nobreak
\nobreak\extramarks{#1}{}\nobreak
}

\setcounter{secnumdepth}{0} % Removes default section numbers
\newcounter{homeworkProblemCounter} % Creates a counter to keep track of the number of problems

\newcommand{\homeworkProblemName}{}
\newenvironment{homeworkProblem}[1][Problem \arabic{homeworkProblemCounter}]{ % Makes a new environment called homeworkProblem which takes 1 argument (custom name) but the default is "Problem #"
\stepcounter{homeworkProblemCounter} % Increase counter for number of problems
\renewcommand{\homeworkProblemName}{#1} % Assign \homeworkProblemName the name of the problem
\section{\homeworkProblemName} % Make a section in the document with the custom problem count
\enterProblemHeader{\homeworkProblemName} % Header and footer within the environment
}{
\exitProblemHeader{\homeworkProblemName} % Header and footer after the environment
}

\newcommand{\problemAnswer}[1]{ % Defines the problem answer command with the content as the only argument
\noindent\framebox[\columnwidth][c]{\begin{minipage}{0.98\columnwidth}#1\end{minipage}} % Makes the box around the problem answer and puts the content inside
}

\newcommand{\homeworkSectionName}{}
\newenvironment{homeworkSection}[1]{ % New environment for sections within homework problems, takes 1 argument - the name of the section
\renewcommand{\homeworkSectionName}{#1} % Assign \homeworkSectionName to the name of the section from the environment argument
\subsection{\homeworkSectionName} % Make a subsection with the custom name of the subsection
\enterProblemHeader{\homeworkProblemName\ [\homeworkSectionName]} % Header and footer within the environment
}{
\enterProblemHeader{\homeworkProblemName} % Header and footer after the environment
}
   
%----------------------------------------------------------------------------------------
%	NAME AND CLASS SECTION
%----------------------------------------------------------------------------------------

\newcommand{\hmwkTitle}{Problem Set\ \#1} % Assignment title
%\newcommand{\hmwkDueDate}{Monday,\ January\ 1,\ 2012} % Due date
\newcommand{\hmwkClass}{CS\ 161} % Course/class
%\newcommand{\hmwkClassTime}{10:30am} % Class/lecture time
\newcommand{\hmwkClassInstructor}{Tim Roughgarden} % Teacher/lecturer
%\newcommand{\hmwkAuthorName}{John Smith} % Your name

%----------------------------------------------------------------------------------------
%	TITLE PAGE
%----------------------------------------------------------------------------------------

\title{
\vspace{2in}
\textmd{\textbf{\hmwkClass:\ \hmwkTitle}}\\
%\normalsize\vspace{0.1in}\small{Due\ on\ \hmwkDueDate}\\
\vspace{0.1in}\large{Instructor: \textit{\hmwkClassInstructor}}
%\vspace{0.1in}\large{\textit{\hmwkClassInstructor\ \hmwkClassTime}}
\vspace{3in}
}

%\author{\textbf{\hmwkAuthorName}}
\date{} % Insert date here if you want it to appear below your name

%----------------------------------------------------------------------------------------

\begin{document}

%\maketitle

%----------------------------------------------------------------------------------------
%	TABLE OF CONTENTS
%----------------------------------------------------------------------------------------

%\setcounter{tocdepth}{1} % Uncomment this line if you don't want subsections listed in the ToC

%\newpage
%\tableofcontents
%\newpage

%%%
%  General stuff
%%%

\begin{homeworkProblem}
We are given as input a set of $n$ requests (e.g., for the use of an auditorium), with a known start time $s_i$ and finish time $t_i$ for each request $i$.
Assume that all start and finish times are distinct.
Two requests \emph{conflict} if they overlap in time --- if one of them starts between the start and finish times of the other.
Our goal is to select a maximum-cardinality subset of the given requests that contains no conflicts.
(For example, given three requests consuming the intervals $[0,3], [2,5]$, and $[4,7]$, we want to return the first and third requests.)
We aim to design a greedy algorithm for this problem with the following form: At each iteration we select a new request $i$, including it in the solution-so-far and deleting from future consideration all requests that conflict with $i$.

Which of the following greedy rules is guaranteed to always compute an optimal solution?
\begin{itemize}
\item At each iteration, pick the remaining request with the fewest number of conflicts with other remaining requests (breaking ties arbitrarily).
\item At each iteration, pick the remaining request which requires the least time (i.e., has the smallest value of $t_i - s_i$) (breaking ties arbitrarily).
\item At each iteration, pick the remaining request with the earliest finish time.
\item At each iteration, pick the remaining request with the earliest start time.
\end{itemize}

\problemAnswer{ % Answer
Let us find counterexamples for the first, second and the last choices.

\begin{itemize}
\item For the second choice, consider the example $[0, 4], [5, 9], [3, 6]$.
If the shortest request is selected at each iteration, then we will end up with $[3, 6]$ after the first iteration.
However, we should have selected $[0, 4]$ and $[3, 6]$.

\item For the last choice, consider the example $[0, 10], [1, 2]$ and $[3, 4]$.
If we pick the remaining request with the earliest start time, the process will finish with $[0, 10]$, but we want $[1, 2]$ and $[3, 4]$.

\item It is a bit tricky for the first choice.
Consider the requests $[0, 2], [4, 6], [8, 10], [12, 14]$, and $[5, 9]$.
The conflicts to other requests are $0, 1, 1, 0$ and $2$, respectively.
To construct a counterexample, we add $x \geq 3$ jobs with starting time $s \in (0,2)$ and ending time $e \in (4,5)$ and $y \geq 3$ jobs with $s \in (9,10)$ and $e \in (12,14)$.
Now at first iteration, we will select request $[5, 9]$ (still only two conflicts) and delete $[4, 6]$ and $[8, 10]$.
However, we would still like to select $[0, 2], [4, 6], [8, 10]$, and $[12, 14]$.
\end{itemize}

For the third choice, suppose we select request $x$ with the earliest finishing time.
Then we may remove many requests $r_i, i = 1, 2, \dots$ that overlap with $x$.
As such, at most 1 request in the set $\{x, r_1, r_2, \dots\}$ can be selected in the optimal solution.
Hence, for every interval in the optimal solution, there is an interval in the greedy solution.
This problem is the interval scheduling problem: \url{https://en.wikipedia.org/wiki/Interval_scheduling}.

}
\end{homeworkProblem}

\newpage
\begin{homeworkProblem}
We are given as input a set of $n$ jobs, where job $j$ has a processing time $p_j$ and a deadline $d_j$.
Recall the definition of completion times $C_j$ from the video lectures.
Given a schedule (i.e., an ordering of the jobs), we define the lateness $l_j$ of job $j$ as the amount of time $C_j - d_j$ after its deadline that the job completes, or as 0 if $C_j \leq d_j$.
Our goal is to minimize the maximum lateness, $\max_{j} l_j$.

Which of the following greedy rules produces an ordering that minimizes the maximum lateness? You can assume that all processing times and deadlines are distinct.
\begin{itemize}
\item None of the other answers are correct.
\item Schedule the requests in increasing order of the product $d_j \cdot p_j$.
\item Schedule the requests in increasing order of processing time $p_j$.
\item Schedule the requests in increasing order of deadline $d_j$.
\end{itemize}

\problemAnswer{
Let us proof the last choice by an exchange argument.

First arrange the $n$ jobs in increasing order of deadlines: $d_1 < d_2 < \cdots < d_n$.
The lateness of job $i$ is defined as:
\begin{align}
l_i =
\begin{cases}
\sum_{k=1}^{i} p_k - d_i, \quad C_i > d_i, \\
0, \quad C_i \leq d_i.
\end{cases}
\end{align}

Suppose we have the optimal solution $\sigma^\ast$ that is better than the greedy solution $\sigma$.
In solution $\sigma$, we simply have the sequence $1, 2, \cdots, n$.
Since $\sigma^\ast \neq \sigma$, there are consecutive jobs $i, j$ in $\sigma^\ast$ with $i > j$ but $d_i < d_j$.

Now, swap the ordering of $i$ and $j$.
Consequently, the lateness of any job other than $i$ and $j$ is unaffected;
the lateness of $i$ goes down by $p_j$ (if not zero);
and the lateness of $j$ goes up by $p_i$ (if not zero).
Thus, if the maximum lateness (ML) appears in jobs other than $i$ and $j$, the ML among $n$ jobs is unaffected.
If the ML appears in $i$ or $j$, the ML is $\max(p_j - d_j, p_i + p_j - d_i)$ before swapping and $\max(p_i - d_i, p_i + p_j - d_j)$ after swapping.
Because $p_j > 0$ and $d_i < d_j$, we have
\begin{align}
\begin{cases}
p_i - d_i &< p_i + p_j - d_i, \\
p_i + p_j - d_j &< p_i + p_j - d_i,
\end{cases}
\end{align}
and thus $\max(p_i - d_i, p_i + p_j - d_j) < p_i + p_j - d_i$, indicating that swapping $i$ and $j$ improves the ML.
This is contradict to the optimality of $\sigma^\ast$.
}
\end{homeworkProblem}

\end{document}
